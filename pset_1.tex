% Options for packages loaded elsewhere
\PassOptionsToPackage{unicode}{hyperref}
\PassOptionsToPackage{hyphens}{url}
\PassOptionsToPackage{dvipsnames,svgnames,x11names}{xcolor}
%
\documentclass[
  11pt,
  letterpaper,
  DIV=11,
  numbers=noendperiod]{scrartcl}

\usepackage{amsmath,amssymb}
\usepackage{iftex}
\ifPDFTeX
  \usepackage[T1]{fontenc}
  \usepackage[utf8]{inputenc}
  \usepackage{textcomp} % provide euro and other symbols
\else % if luatex or xetex
  \usepackage{unicode-math}
  \defaultfontfeatures{Scale=MatchLowercase}
  \defaultfontfeatures[\rmfamily]{Ligatures=TeX,Scale=1}
\fi
\usepackage{lmodern}
\ifPDFTeX\else  
    % xetex/luatex font selection
\fi
% Use upquote if available, for straight quotes in verbatim environments
\IfFileExists{upquote.sty}{\usepackage{upquote}}{}
\IfFileExists{microtype.sty}{% use microtype if available
  \usepackage[]{microtype}
  \UseMicrotypeSet[protrusion]{basicmath} % disable protrusion for tt fonts
}{}
\makeatletter
\@ifundefined{KOMAClassName}{% if non-KOMA class
  \IfFileExists{parskip.sty}{%
    \usepackage{parskip}
  }{% else
    \setlength{\parindent}{0pt}
    \setlength{\parskip}{6pt plus 2pt minus 1pt}}
}{% if KOMA class
  \KOMAoptions{parskip=half}}
\makeatother
\usepackage{xcolor}
\usepackage[margin=1in,heightrounded]{geometry}
\setlength{\emergencystretch}{3em} % prevent overfull lines
\setcounter{secnumdepth}{-\maxdimen} % remove section numbering
% Make \paragraph and \subparagraph free-standing
\ifx\paragraph\undefined\else
  \let\oldparagraph\paragraph
  \renewcommand{\paragraph}[1]{\oldparagraph{#1}\mbox{}}
\fi
\ifx\subparagraph\undefined\else
  \let\oldsubparagraph\subparagraph
  \renewcommand{\subparagraph}[1]{\oldsubparagraph{#1}\mbox{}}
\fi

\usepackage{color}
\usepackage{fancyvrb}
\newcommand{\VerbBar}{|}
\newcommand{\VERB}{\Verb[commandchars=\\\{\}]}
\DefineVerbatimEnvironment{Highlighting}{Verbatim}{commandchars=\\\{\}}
% Add ',fontsize=\small' for more characters per line
\newenvironment{Shaded}{}{}
\newcommand{\AlertTok}[1]{\textcolor[rgb]{0.16,0.16,0.16}{\textbf{\colorbox[rgb]{0.80,0.14,0.11}{#1}}}}
\newcommand{\AnnotationTok}[1]{\textcolor[rgb]{0.60,0.59,0.10}{#1}}
\newcommand{\AttributeTok}[1]{\textcolor[rgb]{0.84,0.60,0.13}{#1}}
\newcommand{\BaseNTok}[1]{\textcolor[rgb]{0.96,0.45,0.00}{#1}}
\newcommand{\BuiltInTok}[1]{\textcolor[rgb]{0.84,0.36,0.05}{#1}}
\newcommand{\CharTok}[1]{\textcolor[rgb]{0.69,0.38,0.53}{#1}}
\newcommand{\CommentTok}[1]{\textcolor[rgb]{0.57,0.51,0.45}{#1}}
\newcommand{\CommentVarTok}[1]{\textcolor[rgb]{0.57,0.51,0.45}{#1}}
\newcommand{\ConstantTok}[1]{\textcolor[rgb]{0.69,0.38,0.53}{\textbf{#1}}}
\newcommand{\ControlFlowTok}[1]{\textcolor[rgb]{0.80,0.14,0.11}{\textbf{#1}}}
\newcommand{\DataTypeTok}[1]{\textcolor[rgb]{0.84,0.60,0.13}{#1}}
\newcommand{\DecValTok}[1]{\textcolor[rgb]{0.96,0.45,0.00}{#1}}
\newcommand{\DocumentationTok}[1]{\textcolor[rgb]{0.60,0.59,0.10}{#1}}
\newcommand{\ErrorTok}[1]{\textcolor[rgb]{0.80,0.14,0.11}{\underline{#1}}}
\newcommand{\ExtensionTok}[1]{\textcolor[rgb]{0.41,0.62,0.42}{\textbf{#1}}}
\newcommand{\FloatTok}[1]{\textcolor[rgb]{0.96,0.45,0.00}{#1}}
\newcommand{\FunctionTok}[1]{\textcolor[rgb]{0.41,0.62,0.42}{#1}}
\newcommand{\ImportTok}[1]{\textcolor[rgb]{0.41,0.62,0.42}{#1}}
\newcommand{\InformationTok}[1]{\textcolor[rgb]{0.16,0.16,0.16}{\colorbox[rgb]{0.51,0.65,0.60}{#1}}}
\newcommand{\KeywordTok}[1]{\textcolor[rgb]{0.24,0.22,0.21}{\textbf{#1}}}
\newcommand{\NormalTok}[1]{\textcolor[rgb]{0.24,0.22,0.21}{#1}}
\newcommand{\OperatorTok}[1]{\textcolor[rgb]{0.24,0.22,0.21}{#1}}
\newcommand{\OtherTok}[1]{\textcolor[rgb]{0.41,0.62,0.42}{#1}}
\newcommand{\PreprocessorTok}[1]{\textcolor[rgb]{0.84,0.36,0.05}{#1}}
\newcommand{\RegionMarkerTok}[1]{\textcolor[rgb]{0.57,0.51,0.45}{\colorbox[rgb]{0.98,0.96,0.84}{#1}}}
\newcommand{\SpecialCharTok}[1]{\textcolor[rgb]{0.69,0.38,0.53}{#1}}
\newcommand{\SpecialStringTok}[1]{\textcolor[rgb]{0.60,0.59,0.10}{#1}}
\newcommand{\StringTok}[1]{\textcolor[rgb]{0.60,0.59,0.10}{#1}}
\newcommand{\VariableTok}[1]{\textcolor[rgb]{0.27,0.52,0.53}{#1}}
\newcommand{\VerbatimStringTok}[1]{\textcolor[rgb]{0.60,0.59,0.10}{#1}}
\newcommand{\WarningTok}[1]{\textcolor[rgb]{0.16,0.16,0.16}{\colorbox[rgb]{0.98,0.74,0.18}{#1}}}

\providecommand{\tightlist}{%
  \setlength{\itemsep}{0pt}\setlength{\parskip}{0pt}}\usepackage{longtable,booktabs,array}
\usepackage{calc} % for calculating minipage widths
% Correct order of tables after \paragraph or \subparagraph
\usepackage{etoolbox}
\makeatletter
\patchcmd\longtable{\par}{\if@noskipsec\mbox{}\fi\par}{}{}
\makeatother
% Allow footnotes in longtable head/foot
\IfFileExists{footnotehyper.sty}{\usepackage{footnotehyper}}{\usepackage{footnote}}
\makesavenoteenv{longtable}
\usepackage{graphicx}
\makeatletter
\def\maxwidth{\ifdim\Gin@nat@width>\linewidth\linewidth\else\Gin@nat@width\fi}
\def\maxheight{\ifdim\Gin@nat@height>\textheight\textheight\else\Gin@nat@height\fi}
\makeatother
% Scale images if necessary, so that they will not overflow the page
% margins by default, and it is still possible to overwrite the defaults
% using explicit options in \includegraphics[width, height, ...]{}
\setkeys{Gin}{width=\maxwidth,height=\maxheight,keepaspectratio}
% Set default figure placement to htbp
\makeatletter
\def\fps@figure{htbp}
\makeatother

\newcommand{\var}{\text{Var}}
\newcommand{\cov}{\text{Cov}}
\newcommand{\iid}{\overset{\text{iid}}{\sim}}
\newcommand{\argmax}{\mathop{\mathrm{argmax}}\limits}
\newcommand{\argmin}{\mathop{\mathrm{argmin}}\limits}
\newcommand{\inprob}{\overset  p \to }
\newcommand{\as}{\overset  a.s. \to }
\newcommand{\indist}{\overset  d \to }

\newcommand{\bern}{\text{Bernoulli}}
\newcommand{\hypg}{\text{HyperGeometric}}
\newcommand{\negbin}{\text{NegBin}}
\newcommand{\cauchy}{\text{Cauchy}}
\newcommand{\lap}{\text{Laplace}}
\newcommand{\bin}{\text{Binomial}}
\newcommand{\pois}{\text{Poisson}}
\newcommand{\unif}{\text{Uniform}}
\newcommand{\geom}{\text{Geometric}}
\newcommand{\expn}{\text{Exponential}}
\newcommand{\gam}{\text{Gamma}}
\newcommand{\noga}{\text{NormalGamma}}
\newcommand{\invg}{\text{InverseGamma}}
\newcommand{\beeta}{\text{Beta}}
\newcommand{\cat}{\text{Categorical}}
\newcommand{\dir}{\text{Dirichlet}}

\usepackage{bm}
\usepackage{cellspace}
\usepackage{multirow}
\KOMAoption{captions}{tableheading}
\makeatletter
\@ifpackageloaded{tcolorbox}{}{\usepackage[skins,breakable]{tcolorbox}}
\@ifpackageloaded{fontawesome5}{}{\usepackage{fontawesome5}}
\definecolor{quarto-callout-color}{HTML}{909090}
\definecolor{quarto-callout-note-color}{HTML}{0758E5}
\definecolor{quarto-callout-important-color}{HTML}{CC1914}
\definecolor{quarto-callout-warning-color}{HTML}{EB9113}
\definecolor{quarto-callout-tip-color}{HTML}{00A047}
\definecolor{quarto-callout-caution-color}{HTML}{FC5300}
\definecolor{quarto-callout-color-frame}{HTML}{acacac}
\definecolor{quarto-callout-note-color-frame}{HTML}{4582ec}
\definecolor{quarto-callout-important-color-frame}{HTML}{d9534f}
\definecolor{quarto-callout-warning-color-frame}{HTML}{f0ad4e}
\definecolor{quarto-callout-tip-color-frame}{HTML}{02b875}
\definecolor{quarto-callout-caution-color-frame}{HTML}{fd7e14}
\makeatother
\makeatletter
\makeatother
\makeatletter
\makeatother
\makeatletter
\@ifpackageloaded{caption}{}{\usepackage{caption}}
\AtBeginDocument{%
\ifdefined\contentsname
  \renewcommand*\contentsname{Table of contents}
\else
  \newcommand\contentsname{Table of contents}
\fi
\ifdefined\listfigurename
  \renewcommand*\listfigurename{List of Figures}
\else
  \newcommand\listfigurename{List of Figures}
\fi
\ifdefined\listtablename
  \renewcommand*\listtablename{List of Tables}
\else
  \newcommand\listtablename{List of Tables}
\fi
\ifdefined\figurename
  \renewcommand*\figurename{Figure}
\else
  \newcommand\figurename{Figure}
\fi
\ifdefined\tablename
  \renewcommand*\tablename{Table}
\else
  \newcommand\tablename{Table}
\fi
}
\@ifpackageloaded{float}{}{\usepackage{float}}
\floatstyle{ruled}
\@ifundefined{c@chapter}{\newfloat{codelisting}{h}{lop}}{\newfloat{codelisting}{h}{lop}[chapter]}
\floatname{codelisting}{Listing}
\newcommand*\listoflistings{\listof{codelisting}{List of Listings}}
\makeatother
\makeatletter
\@ifpackageloaded{caption}{}{\usepackage{caption}}
\@ifpackageloaded{subcaption}{}{\usepackage{subcaption}}
\makeatother
\makeatletter
\makeatother
\ifLuaTeX
  \usepackage{selnolig}  % disable illegal ligatures
\fi
\IfFileExists{bookmark.sty}{\usepackage{bookmark}}{\usepackage{hyperref}}
\IfFileExists{xurl.sty}{\usepackage{xurl}}{} % add URL line breaks if available
\urlstyle{same} % disable monospaced font for URLs
\hypersetup{
  pdftitle={Problem Set \#1},
  pdfauthor={Luke Benz},
  colorlinks=true,
  linkcolor={blue},
  filecolor={Maroon},
  citecolor={Blue},
  urlcolor={Blue},
  pdfcreator={LaTeX via pandoc}}

\title{Problem Set \#1}
\usepackage{etoolbox}
\makeatletter
\providecommand{\subtitle}[1]{% add subtitle to \maketitle
  \apptocmd{\@title}{\par {\large #1 \par}}{}{}
}
\makeatother
\subtitle{BST 258: Causal Inference -- Theory and Practice}
\author{Luke Benz}
\date{2024-02-09}

\begin{document}
\maketitle
\begin{tcolorbox}[enhanced jigsaw, arc=.35mm, colback=white, bottomrule=.15mm, breakable, opacityback=0, rightrule=.15mm, toprule=.15mm, leftrule=.75mm, left=2mm, colframe=quarto-callout-note-color-frame]

\textbf{GitHub URL}\vspace{2mm}

Please find my GitHub reporistory at
https://github.com/lbenz730/bst258\_pset1

\end{tcolorbox}

\hypertarget{question-1}{%
\subsection{Question 1}\label{question-1}}

Thank you for these resources.

\hypertarget{question-2}{%
\subsection{Question 2}\label{question-2}}

\hypertarget{a}{%
\subsubsection{2a)}\label{a}}

\begin{tcolorbox}[enhanced jigsaw, arc=.35mm, colback=white, bottomrule=.15mm, breakable, opacityback=0, rightrule=.15mm, toprule=.15mm, leftrule=.75mm, left=2mm, colframe=quarto-callout-note-color-frame]

\textbf{Answer}\vspace{2mm}

\[
\begin{aligned}
P(A = 1) &= \frac{m}{n} \\
P(A = 0) &= \frac{n-m}{n} = 1 - \frac{m}{n} \\
\end{aligned}
\]

\end{tcolorbox}

\hypertarget{b}{%
\subsubsection{2b)}\label{b}}

Note that \(P(A_i, A_j) = P(A_i)P(A_j|A_i)\). The latter probability
just needs to account for the fact there is one fewer treatment
assignment slot in the group of \(A_i\), and also 1 fewer subject to
randomize. Using this, we complete the following contingency table to
specify the joing distribution of \((A_i, A_j)\) for \(i \neq j\)

\begin{tcolorbox}[enhanced jigsaw, arc=.35mm, colback=white, bottomrule=.15mm, breakable, opacityback=0, rightrule=.15mm, toprule=.15mm, leftrule=.75mm, left=2mm, colframe=quarto-callout-note-color-frame]

\textbf{Answer}\vspace{2mm}

\begin{table}[h]
\centering
\begin{tabular}{c|c|c|c|c}
$A_i$ & $A_j$ & $P(A_i)$ & $P(A_j|A_i)$ & $P(A_i, P_j)$ \\
\hline
1 & 1 & $\frac{m}{n}$ & $\frac{m-1}{n-1}$ & $\frac{m(m-1)}{n(n-1)}$ \\
1 & 0 & $\frac{m}{n}$ & $\frac{n-m}{n-1}$ & $\frac{m(n-m)}{n(n-1)}$ \\
0 & 1 & $\frac{n-m}{n}$ & $\frac{m}{n-1}$ & $\frac{m(n-m)}{n(n-1)}$ \\
0 & 0 & $\frac{n-m}{n}$ & $\frac{n-m-1}{n-1}$ & $\frac{(n-m)(n-m-1)}{n(n-1)}$ 
\end{tabular}
\end{table}

\end{tcolorbox}

\hypertarget{c}{%
\subsubsection{2c)}\label{c}}

\begin{tcolorbox}[enhanced jigsaw, arc=.35mm, colback=white, bottomrule=.15mm, breakable, opacityback=0, rightrule=.15mm, toprule=.15mm, leftrule=.75mm, left=2mm, colframe=quarto-callout-note-color-frame]

\textbf{Variance}\vspace{2mm}

Note that since \(A_i\) is binary
\(\mathbb{E}[A_i^2] = \mathbb{E}[A_i]\). Then we have

\[
\begin{aligned}
\var(A_i) &= \mathbb{E}[A_i^2] - \mathbb{E}[A_i]^2 \\ 
&= \mathbb{E}[A_i] - \mathbb{E}[A_i]^2 \\ 
&= \frac{m}{n} - \frac{m^2}{n^2} \\ 
&= \frac{m(n-m)}{n^2}
\end{aligned}
\]

\end{tcolorbox}

\begin{tcolorbox}[enhanced jigsaw, arc=.35mm, colback=white, bottomrule=.15mm, breakable, opacityback=0, rightrule=.15mm, toprule=.15mm, leftrule=.75mm, left=2mm, colframe=quarto-callout-note-color-frame]

\textbf{Covariance}\vspace{2mm}

For the covariance calculation, note that
\(\mathbb{E}[A_iA_j] = P(A_i = 1, A_j = 1) = \frac{m(m-1)}{n(n-1)}\)
from above.

\[
\begin{aligned}
\cov(A_i, A_j) &= \mathbb{E}[A_iA_j] - \mathbb{E}[A_i]\mathbb{E}[A_j] \\ 
&= \frac{m(m-1)}{n(n-1)} - \frac{m^2}{n^2} \\ 
&= \frac{nm(m-1) - m^2(n-1)}{n^2(n-1)} \\ 
&= \frac{nm^2 - nm - nm^2 + m^2}{n^2(n-1)} \\
&= \frac{m(m-n)}{n^2(n-1)} \\
\end{aligned}
\]

\end{tcolorbox}

\hypertarget{d}{%
\subsubsection{2d)}\label{d}}

\begin{tcolorbox}[enhanced jigsaw, arc=.35mm, colback=white, bottomrule=.15mm, breakable, opacityback=0, rightrule=.15mm, toprule=.15mm, leftrule=.75mm, left=2mm, colframe=quarto-callout-note-color-frame]

\textbf{Answer}\vspace{2mm}

\[
\begin{aligned}
\mathbb{E}\bigr[\theta^\text{ATT}\bigr] &= \mathbb{E}\biggr[\frac{1}{m}\sum_{i = 1}^n A_i(Y_i^1 - Y_i^0)\biggr] \\
&= \frac{1}{m}\sum_{i = 1}^m \mathbb{E}[A_i(Y_i^1 - Y_i^0)] \\
&= \frac{1}{m}\sum_{i = 1}^m (Y_i^1 - Y_i^0)\mathbb{E}[A_i] \\
&= \frac{1}{m}\sum_{i = 1}^m (Y_i^1 - Y_i^0)\times\frac{m}{n} \\
&= \frac{1}{n}\sum_{i = 1}^m (Y_i^1 - Y_i^0) \\
&= \theta^\text{ATE}
\end{aligned}
\] In expectation, the sample ATT equals the sample ATE.

\end{tcolorbox}

\hypertarget{question-3}{%
\subsection{Question 3}\label{question-3}}

First, we establish the following identities \[
\begin{aligned}
\mathbb{E}[Y_i(1)] &= \frac{1}{n}\sum_{i = 1}^n Y_i(1) \\
&= \frac{1}{n}\sum_{i = 1}^n \biggr(Y_i(0) + \theta\biggr) \\
&= \mathbb{E}[Y_i(0)] + \theta
\end{aligned}
\]

\[
\begin{aligned}
\mathbb{E}[Y_i(1)^2] &= \frac{1}{n}\sum_{i = 1}^n Y_i(1)^2 \\
&= \frac{1}{n}\sum_{i = 1}^n \biggr(Y_i(0) + \theta\biggr)^2 \\
&= \frac{1}{n}\sum_{i = 1}^n \biggr(Y_i(0)^2 + 2\theta Y_i(0) +  \theta^2\biggr) \\
&= \mathbb{E}[Y_i(0)^2] +  2\theta\mathbb{E}[Y_i(0)] + \theta^2
\end{aligned}
\]

\[
\begin{aligned}
\mathbb{E}[Y_i(0)Y_i(1)] &= \sum_{i = 1}^n Y_i(0)Y_i(1) \\
&= \sum_{i = 1}^n [Y_i(1) - \theta]Y_i(1) \\
&= \sum_{i = 1}^n \biggr(Y_i(1)^2 - \theta Y_i(1)\biggr) \\
&= \mathbb{E}[Y_i(1)^2] - \theta\mathbb{E}[Y_i(1)]
\end{aligned}
\]

\begin{tcolorbox}[enhanced jigsaw, arc=.35mm, colback=white, bottomrule=.15mm, breakable, opacityback=0, rightrule=.15mm, toprule=.15mm, leftrule=.75mm, left=2mm, colframe=quarto-callout-note-color-frame]

\textbf{Answer}\vspace{2mm}

Putting these together we have that

\[
\begin{aligned}
\var[Y_i(1)] &= \mathbb{E}[Y_i(1)^2] - \mathbb{E}[Y_i(1)]^2 \\
&= \mathbb{E}[Y_i(0)^2] +  2\theta\mathbb{E}[Y_i(0)] + \theta^2 - \biggr(\mathbb{E}[Y_i(0)] + \theta\biggr)^2 \\
&= \mathbb{E}[Y_i(0)^2] +  2\theta\mathbb{E}[Y_i(0)] + \theta^2 - \biggr(\mathbb{E}[Y_i(0)]^2 +  2\theta\mathbb{E}[Y_i(0)]\theta^2\biggr) \\
&= \mathbb{E}[Y_i(0)^2]- \mathbb{E}[Y_i(0)]^2 \\
&= \var[Y_i(0)]
\end{aligned}
\]

\[
\begin{aligned}
\cov\biggr(Y_i(0), Y_i(1)\biggr) &= \mathbb{E}[Y_i(0)Y_i(1)] - \mathbb{E}[Y_i(0)]\mathbb{E}[Y_i(1)] \\
&= \mathbb{E}[Y_i(1)^2] - \theta\mathbb{E}[Y_i(1)] - \biggr(\mathbb{E}[Y_i(1)] - \theta\biggr)\mathbb{E}[Y_i(1)] \\
&= \mathbb{E}[Y_i(1)^2] - \mathbb{E}[Y_i(1)]^2 \\
&= \var[Y_i(1)]
\end{aligned}
\] Thus we have that \[
\begin{aligned}
\rho\biggr(Y_i(1), Y_i(0)\biggr) &= \frac{\cov\biggr(Y_i(0), Y_i(1)\biggr)}{\sqrt{\var[Y_i(0)]\var[Y_i(1)]}} \\
&= \frac{\var[Y_i(1)]}{\sqrt{\var[Y_i(1)]\var[Y_i(1)]}} \\
&= \frac{\var[Y_i(1)]}{\var[Y_i(1)]} \\
&= 1
\end{aligned}
\]

\end{tcolorbox}

\hypertarget{question-4}{%
\subsection{Question 4}\label{question-4}}

We have \(N = 8\) cups of team, \(T = 4\) of which have tea poured first
and \(M = 4\) of which have milk poured first. Under randomly guessing
which cups had tea poured first, the number of correct guesses
\(X \sim \hypg(8, 4, 4)\) with PMF

\[
P(X = x) =\frac{{4 \choose x}{4 \choose 4-x}}{{8 \choose 4}}
\] Plugging in values of \(x \in \{0, 1, 2, 3, 4\}\) we have that

\begin{tcolorbox}[enhanced jigsaw, arc=.35mm, colback=white, bottomrule=.15mm, breakable, opacityback=0, rightrule=.15mm, toprule=.15mm, leftrule=.75mm, left=2mm, colframe=quarto-callout-note-color-frame]

\textbf{Answer}\vspace{2mm}

\begin{table}[h]
\centering
\begin{tabular}{Sc|Sc}
Cups of Tea Correct ($x$) & $P(X = x)$ \\
\hline
0 & $\frac{1}{70}$ \\
1 & $\frac{16}{70}$ \\
2 & $\frac{36}{70}$ \\
3 & $\frac{16}{70}$ \\
4 & $\frac{1}{70}$ 
\end{tabular}
\end{table}

\end{tcolorbox}

\hypertarget{question-5}{%
\subsection{Question 5}\label{question-5}}

\hypertarget{a-1}{%
\subsubsection{5a)}\label{a-1}}

\begin{tcolorbox}[enhanced jigsaw, arc=.35mm, colback=white, bottomrule=.15mm, breakable, opacityback=0, rightrule=.15mm, toprule=.15mm, leftrule=.75mm, left=2mm, colframe=quarto-callout-note-color-frame]

\textbf{Answer}\vspace{2mm}

We notice that despite treatment \(A\) being superior to treatment \(B\)
within each stone size strata, the overall success rate is higher for
treatment \(B\). Upon closer examination, we notice that a much larger
percentage of subjects on treatment \(B\) have small stones than on
treatment \(A\). Furthermore, we notice that for both treatments, small
stones have higher success rates than do larger stones.

It's clear that these patients were not randomized to treatment
assignment (otherwise we'd expect closer balance of stone size between
arms). It could be the case that treatment \(B\) is the standard course
of action for those presenting with small stones, while treatment \(A\)
is more likely given to patients with large stones.

It could also be the case that there are other factors not shown in
Table 1 which impact both treatment assignments and outcome success
rate. Perhaps for example men are more likely to get large stones and
also treatment \(A\) (maybe pregnant women are exlcuded from treatment
\(A\)), and it's easier to treat stones in men then women. This is
entirely hypothetical, but shows that additional covariates which affect
both treatment AND outcome are important to keep in mind for explaining
the discrepancy in our data.

\end{tcolorbox}

\hypertarget{b-1}{%
\subsubsection{5b)}\label{b-1}}

\begin{tcolorbox}[enhanced jigsaw, arc=.35mm, colback=white, bottomrule=.15mm, breakable, opacityback=0, rightrule=.15mm, toprule=.15mm, leftrule=.75mm, left=2mm, colframe=quarto-callout-note-color-frame]

\textbf{Answer}\vspace{2mm}

\begin{table}[h]
\centering
\begin{tabular}{|Sc|Sc|Sc|Sc|}
\multicolumn{1}{l}{\textbf{Stone Size}} & \multicolumn{1}{l}{\textbf{Gender}} & \multicolumn{1}{l}{\textbf{Treatment $A$}} & \multicolumn{1}{l}{\textbf{Treatment $B$}} \\ 
\hline
\multirow{3}{*}{\centering Small} & Male   & 94\% (79/84)   & 95\% (95/100)  \\
& Female & 67\% (2/3)     & 82\% (139/170) \\
& Both   & 93\% (81/87)   & 87\% (234/270) \\
\hline
\multirow{3}{*}{\centering Large} & Male   & 81\% (100/123) & 100 \% (4/4)   \\
& Female & 66\% (92/140)  & 67\% (51/76)   \\
& Both   & 73\% (192/263) & 69\% (55/80)   \\
\hline
\multirow{3}{*}{\centering Both}  & Male   & 87\% (179/207) & 95\% (99/104)  \\
& Female & 66\% (94/143)  & 77\% (190/246) \\
& Both   & 78\% (273/350) & 83\% (289/350) \\
\hline                       
\end{tabular}
\end{table}

\end{tcolorbox}

\hypertarget{c-1}{%
\subsubsection{5c)}\label{c-1}}

\begin{tcolorbox}[enhanced jigsaw, arc=.35mm, colback=white, bottomrule=.15mm, breakable, opacityback=0, rightrule=.15mm, toprule=.15mm, leftrule=.75mm, left=2mm, colframe=quarto-callout-note-color-frame]

\textbf{Answer}\vspace{2mm}

Some people may refer to this as Simpson's paradox, but it's also an
example of confounding. What's important is that we can't readily
compare two treatments as is (i.e.~without considering other factors) if
data are not randomized. Moreover, our colleague stratified by gender,
but had they stratified by yet another variable, perhaps the effect
would've flipped again. In other words, there is no knowledge of all
variables that confounding the treament-outcome relationship. In
general, the implication is that confounding is hard to deal with and we
must take care in addressing the potential of confounding when comparing
outcomes between various treatments.

\end{tcolorbox}

\hypertarget{question-6}{%
\subsection{Question 6}\label{question-6}}

First, I'll set up some useful functions.

\begin{Shaded}
\begin{Highlighting}[]
\DocumentationTok{\#\#\# Function to sample data}
\DocumentationTok{\#\#\#}
\DocumentationTok{\#\# n = n0 + n1}
\DocumentationTok{\#\# mu = vector of group means}
\DocumentationTok{\#\# sigma2 = vector of group variances}
\DocumentationTok{\#\# p = prob[A = 1]}
\NormalTok{sample\_data }\OtherTok{\textless{}{-}} \ControlFlowTok{function}\NormalTok{(n, mu, sigma2, p) \{}
\NormalTok{  Y0 }\OtherTok{\textless{}{-}} \FunctionTok{rnorm}\NormalTok{(n, mu[}\DecValTok{1}\NormalTok{], }\FunctionTok{sqrt}\NormalTok{(sigma2[}\DecValTok{1}\NormalTok{]))}
\NormalTok{  Y1 }\OtherTok{\textless{}{-}} \FunctionTok{rnorm}\NormalTok{(n, mu[}\DecValTok{2}\NormalTok{], }\FunctionTok{sqrt}\NormalTok{(sigma2[}\DecValTok{2}\NormalTok{]))}
\NormalTok{  A }\OtherTok{\textless{}{-}} \FunctionTok{rbinom}\NormalTok{(n, }\DecValTok{1}\NormalTok{, p)}
\NormalTok{  Y }\OtherTok{\textless{}{-}} \FunctionTok{ifelse}\NormalTok{(A }\SpecialCharTok{==} \DecValTok{1}\NormalTok{, Y1, Y0)}
\NormalTok{  df }\OtherTok{\textless{}{-}} \FunctionTok{tibble}\NormalTok{(}\StringTok{\textquotesingle{}Y\textquotesingle{}} \OtherTok{=}\NormalTok{ Y,}
               \StringTok{\textquotesingle{}A\textquotesingle{}} \OtherTok{=}\NormalTok{ A)}
  \FunctionTok{return}\NormalTok{(df)}
\NormalTok{\}}

\DocumentationTok{\#\#\# Function to run permutation test (returns p{-}value)}
\DocumentationTok{\#\#}
\DocumentationTok{\#\# Y vector of outcomes}
\DocumentationTok{\#\# A vector of treatments}
\DocumentationTok{\#\# B = \# of permutations to run}
\NormalTok{perm\_test }\OtherTok{\textless{}{-}} \ControlFlowTok{function}\NormalTok{(Y, A, B) \{}
  \DocumentationTok{\#\#\# Observed test statistics}
\NormalTok{  t\_obs }\OtherTok{\textless{}{-}} \FunctionTok{abs}\NormalTok{(}\FunctionTok{mean}\NormalTok{(Y[A }\SpecialCharTok{==} \DecValTok{1}\NormalTok{]) }\SpecialCharTok{{-}} \FunctionTok{mean}\NormalTok{(Y[A }\SpecialCharTok{==} \DecValTok{0}\NormalTok{]))}
  
  \DocumentationTok{\#\#\# Null distribution }
\NormalTok{  t\_sim }\OtherTok{\textless{}{-}} \FunctionTok{rep}\NormalTok{(}\ConstantTok{NA}\NormalTok{, B) }
  \ControlFlowTok{for}\NormalTok{(b }\ControlFlowTok{in} \DecValTok{1}\SpecialCharTok{:}\NormalTok{B) \{}
\NormalTok{    A\_sim }\OtherTok{\textless{}{-}} \FunctionTok{sample}\NormalTok{(A) }
\NormalTok{    t\_sim[b] }\OtherTok{\textless{}{-}} \FunctionTok{abs}\NormalTok{(}\FunctionTok{mean}\NormalTok{(Y[A\_sim }\SpecialCharTok{==} \DecValTok{1}\NormalTok{]) }\SpecialCharTok{{-}} \FunctionTok{mean}\NormalTok{(Y[A\_sim }\SpecialCharTok{==} \DecValTok{0}\NormalTok{]))}
\NormalTok{  \}}
  
  \DocumentationTok{\#\#\# 2{-}Sided p{-}value}
\NormalTok{  p\_value }\OtherTok{\textless{}{-}} \FunctionTok{mean}\NormalTok{(t\_sim }\SpecialCharTok{\textgreater{}=}\NormalTok{ t\_obs)}
  
  \FunctionTok{return}\NormalTok{(p\_value)}
\NormalTok{\}}

\DocumentationTok{\#\#\# Function to compute weak/sharp null p{-}value on a simulated dataset}
\DocumentationTok{\#\#\#}
\DocumentationTok{\#\#\# df = sampled dataset of A,Y}
\DocumentationTok{\#\#\# B = permutation test dataset}

\NormalTok{compute\_pvalues }\OtherTok{\textless{}{-}} \ControlFlowTok{function}\NormalTok{(df, B) \{}
  \DocumentationTok{\#\#\# Permutation test (for sharp null)}
\NormalTok{  p\_sharp }\OtherTok{\textless{}{-}} \FunctionTok{perm\_test}\NormalTok{(}\AttributeTok{Y =}\NormalTok{ df}\SpecialCharTok{$}\NormalTok{Y, }\AttributeTok{A =}\NormalTok{ df}\SpecialCharTok{$}\NormalTok{A, }\AttributeTok{B =}\NormalTok{ B)}
  
  \DocumentationTok{\#\#\# Testing Weak Null based on normal approximation}
\NormalTok{  ate\_hat }\OtherTok{\textless{}{-}} \FunctionTok{mean}\NormalTok{(df}\SpecialCharTok{$}\NormalTok{Y[df}\SpecialCharTok{$}\NormalTok{A }\SpecialCharTok{==} \DecValTok{1}\NormalTok{]) }\SpecialCharTok{{-}} \FunctionTok{mean}\NormalTok{(df}\SpecialCharTok{$}\NormalTok{Y[df}\SpecialCharTok{$}\NormalTok{A }\SpecialCharTok{==} \DecValTok{0}\NormalTok{]) }
\NormalTok{  var\_hat }\OtherTok{\textless{}{-}} \FunctionTok{var}\NormalTok{(df}\SpecialCharTok{$}\NormalTok{Y[df}\SpecialCharTok{$}\NormalTok{A }\SpecialCharTok{==} \DecValTok{1}\NormalTok{])}\SpecialCharTok{/}\FunctionTok{sum}\NormalTok{(df}\SpecialCharTok{$}\NormalTok{A }\SpecialCharTok{==} \DecValTok{1}\NormalTok{) }\SpecialCharTok{+} \FunctionTok{var}\NormalTok{(df}\SpecialCharTok{$}\NormalTok{Y[df}\SpecialCharTok{$}\NormalTok{A }\SpecialCharTok{==} \DecValTok{1}\NormalTok{])}\SpecialCharTok{/}\FunctionTok{sum}\NormalTok{(df}\SpecialCharTok{$}\NormalTok{A }\SpecialCharTok{==} \DecValTok{0}\NormalTok{)}
\NormalTok{  p\_weak }\OtherTok{\textless{}{-}} \DecValTok{2} \SpecialCharTok{*} \FunctionTok{pnorm}\NormalTok{(}\AttributeTok{q =} \SpecialCharTok{{-}}\FunctionTok{abs}\NormalTok{(ate\_hat), }\AttributeTok{mean =} \DecValTok{0}\NormalTok{, }\AttributeTok{sd =} \FunctionTok{sqrt}\NormalTok{(var\_hat), }\AttributeTok{lower.tail =}\NormalTok{ T)}
  
  \FunctionTok{return}\NormalTok{(}\FunctionTok{list}\NormalTok{(}\StringTok{\textquotesingle{}p\_weak\textquotesingle{}} \OtherTok{=}\NormalTok{ p\_weak, }\StringTok{\textquotesingle{}p\_sharp\textquotesingle{}} \OtherTok{=}\NormalTok{ p\_sharp))}
\NormalTok{\}}
\end{Highlighting}
\end{Shaded}

Next, we'll run the simulation, leveraging parallel processing from the
\texttt{furrr} package in \texttt{R}.

\begin{Shaded}
\begin{Highlighting}[]
\FunctionTok{library}\NormalTok{(tidyverse)}
\FunctionTok{library}\NormalTok{(furrr)}
\FunctionTok{plan}\NormalTok{(}\FunctionTok{multisession}\NormalTok{(}\AttributeTok{workers =}\NormalTok{ parallel}\SpecialCharTok{::}\FunctionTok{detectCores}\NormalTok{() }\SpecialCharTok{{-}} \DecValTok{1}\NormalTok{))}

\DocumentationTok{\#\#\# Set simulation parameters}
\FunctionTok{set.seed}\NormalTok{(}\DecValTok{73097}\NormalTok{)}
\NormalTok{n1 }\OtherTok{\textless{}{-}} \FunctionTok{c}\NormalTok{(}\DecValTok{10}\NormalTok{, }\DecValTok{25}\NormalTok{, }\DecValTok{50}\NormalTok{, }\DecValTok{100}\NormalTok{, }\DecValTok{250}\NormalTok{) }\DocumentationTok{\#\#\# n1 and n0}
\NormalTok{n0 }\OtherTok{\textless{}{-}} \FunctionTok{c}\NormalTok{(}\DecValTok{10}\NormalTok{, }\DecValTok{25}\NormalTok{, }\DecValTok{50}\NormalTok{, }\DecValTok{100}\NormalTok{, }\DecValTok{250}\NormalTok{) }\DocumentationTok{\#\#\# n1 and n0}
\NormalTok{n }\OtherTok{\textless{}{-}}\NormalTok{ n1 }\SpecialCharTok{+}\NormalTok{ n0}
\NormalTok{n\_sims }\OtherTok{\textless{}{-}} \DecValTok{1000}

\DocumentationTok{\#\#\# Run simulation}
\NormalTok{df\_power }\OtherTok{\textless{}{-}} 
  \FunctionTok{tibble}\NormalTok{(}\StringTok{\textquotesingle{}n\textquotesingle{}} \OtherTok{=}\NormalTok{ n,}
         \StringTok{\textquotesingle{}power\_weak\textquotesingle{}} \OtherTok{=} \ConstantTok{NA}\NormalTok{,}
         \StringTok{\textquotesingle{}power\_sharp\textquotesingle{}} \OtherTok{=} \ConstantTok{NA}\NormalTok{)}


\ControlFlowTok{for}\NormalTok{(i }\ControlFlowTok{in} \DecValTok{1}\SpecialCharTok{:}\FunctionTok{length}\NormalTok{(n)) \{}
  \DocumentationTok{\#\#\# Simulate all datasets in one go}
\NormalTok{  simulated\_dfs }\OtherTok{\textless{}{-}} 
    \FunctionTok{map}\NormalTok{(}\DecValTok{1}\SpecialCharTok{:}\NormalTok{n\_sims, }\SpecialCharTok{\textasciitilde{}}\NormalTok{\{}
      \FunctionTok{sample\_data}\NormalTok{(}\AttributeTok{n =}\NormalTok{ n[i], }\AttributeTok{mu =} \FunctionTok{c}\NormalTok{(}\DecValTok{0}\NormalTok{, }\DecValTok{1}\SpecialCharTok{/}\DecValTok{10}\NormalTok{), }\AttributeTok{sigma2 =} \FunctionTok{c}\NormalTok{(}\DecValTok{1}\SpecialCharTok{/}\DecValTok{16}\NormalTok{, }\DecValTok{1}\SpecialCharTok{/}\DecValTok{16}\NormalTok{), }\AttributeTok{p =} \FloatTok{0.5}\NormalTok{)}
\NormalTok{    \})}
  
  \DocumentationTok{\#\#\# Get all p\_values in one go}
\NormalTok{  df\_pvalues }\OtherTok{\textless{}{-}} 
    \FunctionTok{future\_map\_dfr}\NormalTok{(simulated\_dfs, }\SpecialCharTok{\textasciitilde{}}\FunctionTok{compute\_pvalues}\NormalTok{(}\AttributeTok{df =}\NormalTok{ .x, }\AttributeTok{B =} \DecValTok{10000}\NormalTok{),}
                   \AttributeTok{.options =} \FunctionTok{furrr\_options}\NormalTok{(}\AttributeTok{seed =} \DecValTok{73091}\NormalTok{))}
  
  \DocumentationTok{\#\#\# Compute Power}
\NormalTok{  df\_power}\SpecialCharTok{$}\NormalTok{power\_sharp[i] }\OtherTok{\textless{}{-}} \FunctionTok{mean}\NormalTok{(df\_pvalues}\SpecialCharTok{$}\NormalTok{p\_sharp }\SpecialCharTok{\textless{}=} \FloatTok{0.05}\NormalTok{)}
\NormalTok{  df\_power}\SpecialCharTok{$}\NormalTok{power\_weak[i] }\OtherTok{\textless{}{-}} \FunctionTok{mean}\NormalTok{(df\_pvalues}\SpecialCharTok{$}\NormalTok{p\_weak }\SpecialCharTok{\textless{}=} \FloatTok{0.05}\NormalTok{)}
\NormalTok{\}}
\end{Highlighting}
\end{Shaded}

Finally, we plot the results.

\begin{Shaded}
\begin{Highlighting}[]
\NormalTok{df\_plot }\OtherTok{\textless{}{-}} 
\NormalTok{  df\_power }\SpecialCharTok{\%\textgreater{}\%} 
  \FunctionTok{pivot\_longer}\NormalTok{(}\AttributeTok{cols =} \FunctionTok{contains}\NormalTok{(}\StringTok{\textquotesingle{}power\textquotesingle{}}\NormalTok{),}
               \AttributeTok{names\_to =} \StringTok{\textquotesingle{}hypothesis\textquotesingle{}}\NormalTok{,}
               \AttributeTok{values\_to =} \StringTok{\textquotesingle{}power\textquotesingle{}}\NormalTok{,}
               \AttributeTok{names\_prefix =} \StringTok{\textquotesingle{}power\_\textquotesingle{}}\NormalTok{) }\SpecialCharTok{\%\textgreater{}\%} 
  \FunctionTok{mutate}\NormalTok{(}\StringTok{\textquotesingle{}hypothesis\textquotesingle{}} \OtherTok{=} \FunctionTok{paste}\NormalTok{(tools}\SpecialCharTok{::}\FunctionTok{toTitleCase}\NormalTok{(hypothesis), }\StringTok{\textquotesingle{}Null\textquotesingle{}}\NormalTok{))}

\FunctionTok{ggplot}\NormalTok{(df\_plot, }\FunctionTok{aes}\NormalTok{(}\AttributeTok{x =}\NormalTok{ n, }\AttributeTok{y =}\NormalTok{ power)) }\SpecialCharTok{+} 
  \FunctionTok{geom\_point}\NormalTok{(}\FunctionTok{aes}\NormalTok{(}\AttributeTok{color =}\NormalTok{ hypothesis)) }\SpecialCharTok{+} 
  \FunctionTok{geom\_line}\NormalTok{(}\FunctionTok{aes}\NormalTok{(}\AttributeTok{color =}\NormalTok{ hypothesis)) }\SpecialCharTok{+} 
  \FunctionTok{labs}\NormalTok{(}\AttributeTok{x =} \StringTok{\textquotesingle{}n\textquotesingle{}}\NormalTok{,}
       \AttributeTok{y =} \StringTok{\textquotesingle{}Power\textquotesingle{}}\NormalTok{,}
       \AttributeTok{color =} \StringTok{\textquotesingle{}Hypothesis\textquotesingle{}}\NormalTok{,}
       \AttributeTok{title =} \StringTok{\textquotesingle{}Comparison of Power Between}\SpecialCharTok{\textbackslash{}n}\StringTok{Weak and Sharp Null\textquotesingle{}}\NormalTok{)}
\end{Highlighting}
\end{Shaded}

\begin{figure}[H]

{\centering \includegraphics{pset_1_files/figure-pdf/unnamed-chunk-4-1.pdf}

}

\end{figure}

\begin{tcolorbox}[enhanced jigsaw, arc=.35mm, colback=white, bottomrule=.15mm, breakable, opacityback=0, rightrule=.15mm, toprule=.15mm, leftrule=.75mm, left=2mm, colframe=quarto-callout-note-color-frame]

\textbf{Answer Summary}\vspace{2mm}

Because Fisher's Null implies Neyman's Null, we might hyopothesize that
rejection of \(H_0^\text{weak}\) implies rejection of
\(H_0^\text{sharp}\). However, we see via simulations that this is not
the case. For small values of \(n\) the power against
\(H_0^\text{weak}\) is actually slightly higher than
\(H_0^\text{sharp}\), suggesting that evidence against the weak null
need not imply evidence against the sharp null.For larger values of
\(n\), the power of these two tests is essentially identical.

At first, this result seems somewhat paradoxical. One possible
explanation may be that the permutation test is inefficient in small
samples, relative to using the normal approximation to test the weak
null, even with a conservative variance estimator. Overall, this
simulation shows the importance of testing out ideas because our
intuition can sometimes lead us astray.

\end{tcolorbox}

\hypertarget{question-7}{%
\subsection{Question 7}\label{question-7}}

\hypertarget{a-2}{%
\subsubsection{7a)}\label{a-2}}

We begin by differentiating with respect to both \(\alpha, \beta\) and
setting the two equations equal to 0 \[
\begin{aligned}
\frac{\partial}{\partial\alpha}\biggr[\frac{1}{2n}\sum_{i = 1}^n (Y_i - \alpha - \beta A_i)^2\biggr] &= -\frac{1}{n}\sum_{i = 1}^n(Y_i - \alpha - \beta A_i) = 0 \\
\frac{\partial}{\partial\beta}\biggr[\frac{1}{2n}\sum_{i = 1}^n(Y_i - \alpha - \beta A_i)^2\biggr] &= -\frac{1}{n}\sum_{i = 1}^n A_i(Y_i - \alpha - \beta A_i) = 0
\end{aligned}
\] Since \(\bar A = \frac{m}{n}\), the first equation implies that \[
\bar Y - \alpha - \beta\bar A = 0 \implies \alpha = \bar Y - \frac{m}{n}\beta
\]

Substituting this equation into the second equation we have that

\[
\begin{aligned}
0 &= \frac{m}{n} \bar Y_1- \frac{m}{n}\alpha - \frac{m}{n}\beta \implies \\ 
0 &= \bar Y_1- \alpha - \beta \implies \\ 
0 &= \bar Y_1 - \biggr(\bar Y - \frac{m}{n}\beta\biggr) - \beta \implies \\
\frac{n-m}{n}\beta &= \sum_{i = 1}^n A_i\frac{Y_i}{m} - \frac{Y_i}{n} \implies \\
\frac{n-m}{n}\beta &= \sum_{i = 1}^n \frac{A_iY_i}{m} - \frac{A_iY_i + (1-A_i)Y_i}{n} \implies \\
\frac{n-m}{n}\beta &= \sum_{i = 1}^n \frac{A_iY_i(n-m) -  m(1-A_i)Y_i}{nm} \implies \\
\beta &= \frac{n}{n-m}\sum_{i = 1}^n \frac{A_iY_i(n-m) -  m(1-A_i)Y_i}{nm}\implies  \\
\beta &= \frac{1}{m}\sum_{i = 1}^n A_iY_i - \frac{1}{n-m}\sum_{i = 1}^n(1-A_i)Y_i \implies \\
\beta &= \bar Y_1 - \bar Y_0
\end{aligned}
\]

Using this result we then have that \[
\begin{aligned}
\alpha &= \bar Y - \frac{m}{n}\beta \\
&= \bar Y - \frac{m}{n}(\bar Y_1 - \bar Y_0) \\
&= \biggr(\frac{m}{n}\bar Y_1 + \frac{n-m}{n}\bar Y_0\biggr) - \frac{m}{n}(\bar Y_1 - \bar Y_0) \\
&= \bar Y_0
\end{aligned}
\]

\begin{tcolorbox}[enhanced jigsaw, arc=.35mm, colback=white, bottomrule=.15mm, breakable, opacityback=0, rightrule=.15mm, toprule=.15mm, leftrule=.75mm, left=2mm, colframe=quarto-callout-note-color-frame]

\textbf{Answer Summary}\vspace{2mm}

\[
\boxed{
\begin{aligned}
\alpha &= \bar Y_0 \\
\beta &= \bar Y_1 - \bar Y_0
\end{aligned}
}
\]

\end{tcolorbox}

\hypertarget{b-2}{%
\subsubsection{7b)}\label{b-2}}

\begin{tcolorbox}[enhanced jigsaw, arc=.35mm, colback=white, bottomrule=.15mm, breakable, opacityback=0, rightrule=.15mm, toprule=.15mm, leftrule=.75mm, left=2mm, colframe=quarto-callout-note-color-frame]

\textbf{Answer Summary}\vspace{2mm}

Yes, \(\beta\) is a valid estimator for the ATE. It is in fact exactly
the difference in means estimator, which we know from the previous
questions is unbiased for the average treatment effect. This suggests
that under a conditionally randomized expiriment, we can use a saturated
linear model, where the slope is the exact difference in means estimator
which will consistently estimate the ATE.

\end{tcolorbox}



\end{document}
